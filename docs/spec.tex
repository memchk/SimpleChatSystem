\documentclass[12pt]{article}
\usepackage[utf8]{inputenc}
\usepackage[letterpaper, margin=1in]{geometry}
\usepackage{xcolor}
\usepackage[
    colorlinks,
    linkcolor={black!80!black},
    citecolor={blue!50!black},
    urlcolor={blue!80!black}
]{hyperref}
\usepackage{bytefield}
% \usepackage{tabularx}
\usepackage{caption}
% \usepackage{graphicx}
% \usepackage{rotating}
\usepackage{tikz}
\usepackage{float}
\usepackage{titlesec}

\setlength{\parindent}{0em}
\renewcommand{\arraystretch}{1.25}
% Title Block
\title{CS4390.502: Simple Chat System (SCS)}
\author{Carson Page CMP170130}
\date{Spring 2020}
\begin{document}
\begin{titlepage}
    \maketitle
\end{titlepage}
\tableofcontents
\cleardoublepage
\newpage

\setlength{\parskip}{0.5em}

\section{Protocol}
This specification is currently a \textbf{DRAFT}. The protocol format,
semantics, and specification MAY change without incrementing the version number.

\subsection{Overview}
SCS is a networked chat protocol, with a classic client-server architecture. SCS
takes inspiration from pre-existing work like IRC and Web Sockets for the
structure of this document and specification of semantics. However, the
protocol design is my own. As this is an academic
exercise, most of the protocol is novel, but any references will be
properly included within this document.
The action words in this section of the document conform to
\href{https://tools.ietf.org/html/rfc2119}{RFC2119}.

The primary structure is a frame, consisting of a 32 bit length, 16 bit msg
type, and 16 bit Frame Check Sequence (FCS). This is a 8 byte header, and allows for a flexible message
format. The length of a packet MUST be less than 65536 bytes, unless an
extension specifies otherwise. The first bit of the length field indicates
if the packet is the last one in a sequence.

\subsection{Packet Framing}
The packet is framed as shown below. The area marked payload is opcode
dependent, and MUST match that opcodes definition.

\bigskip
\begin{bytefield}[bitwidth=1.1em]{32}
    \bitheader{0-31} \\
    \begin{rightwordgroup}{Header}
        \bitbox{1}{\rotatebox{270}{\small FIN}} &
        \bitbox{3}{RSVD} & \bitbox{28}{Length} \\
        \bitbox{8}{Version} & \bitbox{8}{OpCode} & \bitbox{16}{FCS}
    \end{rightwordgroup} \\
    \wordbox[lr]{1}{Message} \\
    \skippedwords \\
    \wordbox[lrb]{1}{}
\end{bytefield}
\subsection{Version}
The Version field MUST equal 1 (0x0001). Any changes to the SCS protocol that
break backwards compatibility with older implementations that adhere to this
specification MUST have a new version specified.

If an implementation receives a
frame with a version it does not understand, it MUST reply with a
INVALID\_VERSION message and close the connection.
\subsection{FCS}
The FCS is a Cyclic Redundancy Check, 16-bit, polynomial 0x8810 (Koopman)
\href{https://users.ece.cmu.edu/~koopman/crc/c16/0x8810.txt}{[spec]}.
The FCS includes the data in the header and body, except for itself. 
The FCS SHALL be excluded from the calculation entirely. The checksum MUST process the data least significant
bit first, in the order the bytes are received on the wire. The FCS SHALL be
sent on wire in little endian format. 

If an implementation receives a frame with
a bad FCS, it MUST respond with a INVALID\_FCS message and close the connection. This
choice is due to the academic requirement of the FCS, and in all realistic
scenarios the message will be correct due to the underlying stacks abilities.

In future versions of this protocol, this field may be replaced with something
more useful.

\subsection{Valid Entities}
An entity is an object with an identifier capable of sending or receiving
messages. This documents defines channels and users as these entities.
Implementations MAY limit the length of the identifiers used, but MUST NOT limit
it to less than 16 characters.
\subsubsection{User}
A user is an entity that is identified by a username, and represents a singular
actor in SCS. This could be a human user, but also may be the server itself,
various server provided services, or bots. To distinguish users, their username
MUST be prepended on wire with an ASCII dollar sign character (\$). This MAY be
stripped when shown to the user, but must be consistently done so.
\subsubsection{Channel}
A Channels is an entity that is identified by a channel name, and represents a broadcast
channel in SCS. Users MAY join a channel, or be placed in a channel by the
server. To distinguish channels, their channel name MUST be prepended on wire
with an ASCII hash character (\#). This MAY be
stripped when shown to the user, but must be consistently done so. 

\subsection{Messages}
This section enumerates the valid messages of SCS. This document reserves the
first half of the opcode range for the messages defined in this document and for
any official extensions. The top half of the opcode range MAY be used for
third-party or unofficial extensions.

If a message has optional fields, messages defined in this document or future
extensions MUST place those fields after any required fields to help with easier
processing. If a message has one (1) optional field, it may be implicit,
otherwise the message SHOULD have some form of header or delimiter showing which
message fields are included, or have the included fields be unambiguously
specified through the messages semantics. 

Any implementation of SCS that receives a
message with an unknown code MUST respond with a NACK message with optional error
text, then close the connection. Any implementation of SCS that receives an
improperly formatted message MUST respond with a NACK message with optional error text,
then close the connection. This means error checking SHOULD be done locally
before transmitting the message, and bad messages are considered to be hostile.

\subsubsection{Acknowledge (ACK)}
The ACK message contains an opcode of 0x00.

This message contains no additional fields.

\subsubsection{Negative Acknowledge (NACK)}
The NACK message contains an opcode of 0x01.

The client SHOULD display the included error description if available.

\begin{table}[ht]
\begin{tabular} {|l|l|l|}
    \hline 
    Field \# & Type & Description \\
    \hline
    1 & String & Human readable error description (optional) \\
    \hline
\end{tabular}
\end{table}

\subsubsection{Invalid FCS (INVALID\_FCS)}
The INVALID\_FCS message contains an opcode of 0x02.

This message contains no additional fields.

\subsubsection{Invalid Version (INVALID\_VERISON)}
The INVALID\_VERSION message contains an opcode of 0x03.

\begin{table}[ht]
\begin{tabular} {|l|l|l|}
    \hline 
    Field \# & Type & Description \\
    \hline
    1 & Integer & Supported version \\
    \hline
\end{tabular}
\end{table}

\subsubsection{Hello Message (HELO)}
The HELO message contains an opcode of 0x04.

This message SHOULD be sent by any client on connection with a server. A server
MAY terminate a clients connection after a reasonable amount of time if this
message is not sent.

\begin{table}[ht]
\begin{tabular} {|l|l|l|}
    \hline 
    Field \# & Type & Description \\
    \hline
    1 & String & Client Username \\
    2 & String & Initial Channel (Optional) \\
    \hline
\end{tabular}
\end{table}

\subsubsection{Join Channel (JOIN\_CHAN)}
The JOIN\_CHAN message contains an opcode of 0x05.

This message indicates to the server that a user wants access to a channel. This
may require a channel password. If a password is required and is either not
provided or incorrect, the server MUST respond with a NOT\_AUTH message. If
successful, a server SHOULD respond with an ACK message and begin streaming
channel messages to the client. If any other failure occurs, the server SHOULD
respond with a NACK message.

\begin{table}[ht]
\begin{tabular} {|l|l|l|}
    \hline 
    Field \# & Type & Description \\
    \hline
    1 & String & Channel Name \\
    2 & String & Channel Password (Optional) \\
    \hline
\end{tabular}
\end{table}

\subsubsection{Leave Channel (LEAVE\_CHAN)}
The LEAVE\_CHAN message contains an opcode of 0x06.

This message states a client is requesting to leave a channel. If the operation
is successful the server SHOULD send an ACK message, and any streams of channel messages
involving that client and message tuple MUST be terminated. If the operation
fails the server SHOULD send a NACK message.

\begin{table}[ht]
\begin{tabular} {|l|l|l|}
    \hline 
    Field \# & Type & Description \\
    \hline
    1 & String & Channel Name \\
    \hline
\end{tabular}
\end{table}

\subsubsection{Not Authorized (NOT\_AUTH)}
The NOT\_AUTH message contains an opcode of 0x07.

This message SHALL be sent from server to client in response to any operation
from the client where the client has improper authorization to manipulate an
entity. The entity name MUST be given, and no other information must be given to
prevent information leakage. A server implementation MAY log the reason for
failure in a server log, or some out-of-band format like an admin channel for
troubleshooting and intrusion detection.

\begin{table}[ht]
\begin{tabular} {|l|l|l|}
    \hline 
    Field \# & Type & Description \\
    \hline
    1 & String & Entity Name \\
    \hline
\end{tabular}
\end{table}

\subsubsection{Message (MSG)}
The NOT\_AUTH message contains an opcode of 0x08.

This message represents a generic text message sent from one entity to another.

The message will include a single byte MSG\_TYPE field. A user to message MUST
set this field to 0x01. A user to channel message MUST set this field to 0x02. A
channel to user message MUST set this field to 0x03. Remaining values are
reserved unless further specified. If a implementation receives an invalid
MSG\_TYPE, it MUST respond with a NACK and close the connection.

In the case of user to user message, the FROM and TO fields will match the
sender and recipient exactly, and the CHANNEL field will not be included. 

In the case of user to channel messaging, the FROM will include the sending user, and the TO
and CHANNEL names MUST include a channel name and MUST be the same. When the
channel relays the message, it will rewrite the TO field to the subscriber of
the channel, while keeping the FROM and CHANNEL the same.

The server MAY perform verification that the sender entity matches the client
who sent it. The server MAY impose limits of the message content. If these
limits are violated, the server MUST reply with a NACK message with an
appropriate message.

\begin{table}[ht]
\begin{tabular} {|l|l|l|}
    \hline 
    Field \# & Type & Description \\
    \hline
    1 & Byte & Message Type (MSG\_TYPE) \\
    2 & String & FROM \\
    3 & String & TO \\
    4 & String & CHANNEL (MSG\_TYPE 0x02 0x03) \\
    5 & String & Message Content \\
    \hline
\end{tabular}
\end{table}

\subsubsection{File Open (FILE\_Open)}
The FILE\_TRANS message contains an opcode of 0x09.

This message announces to the receiving entity that a file transfer session is
being requested by the FROM entity. A entity MUST
NOT attempt a file transfer to anything other than a user. If this is attempted,
the server MUST respond with a NACK packet and close the connection. If the
receiving entity accepts the file transfer, it must reply with an ACK, otherwise
an NACK message.

\begin{table}[ht]
\begin{tabular} {|l|l|l|}
    \hline 
    Field \# & Type & Description \\
    \hline
    1 & String & FROM \\
    2 & String & TO \\
    3 & String & Filename \\
    4 & Hash & File Hash \\
    \hline
\end{tabular}
\end{table}

\subsubsection{File Transfer (FILE\_TRANS)}
The FILE\_TRANS message contains an opcode of 0x0A.

This message represents a file transfer from one user to another. A entity MUST
NOT attempt a file transfer to anything other than a user. If this is attempted,
the server MUST respond with a NACK packet and close the connection. This
message MUST be sent after a FILE\_OPEN message, and only if a ACK is
replied with first. If this sequence is not followed, the message MAY be discarded.

Once the recipient is done receiving the message, it MUST verify the hash and
either reply with an ACK if the transfer was successful, or NACK if there was an
interruption or corruption.

\begin{table}[ht]
\begin{tabular} {|l|l|l|}
    \hline 
    Field \# & Type & Description \\
    \hline
    1 & String & FROM \\
    2 & String & TO \\
    3 & String & Filename \\
    4 & Hash & File Hash \\
    4 & String & File Content \\
    \hline
\end{tabular}
\end{table}

\subsection{Data Types}
Data types conform to their representation and semantics as defined in C99 unless
otherwise specified here. The on wire representation of the data will be
little-endian unless otherwise specified.
\paragraph{Integers} Integers SHALL be encoded in little-endian format, LSB first.
\paragraph{String} Strings SHALL be composed of a length field, then the data in UTF-8 format. The
length field MUST be 4 bytes.
\paragraph{Hash} This is a cryptographic hash following the SHA-256 format. The
hash should be sent with little-endian byte ordering.
\end{document}